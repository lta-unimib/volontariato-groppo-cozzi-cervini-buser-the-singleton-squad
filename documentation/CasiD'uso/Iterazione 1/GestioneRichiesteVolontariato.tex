\title{GestioneRichiesteVolontariato}
\author{Matteo Cervini}
\date{Due: 27 Dicembre 2024}

\maketitle

\paragraph*{Nota Introduttiva}
\begin{itemize}
    \item \textbf{Disciplina:} Requisiti
    \item \textbf{Due:} 26 Dicembre, 2024 1:00 AM (GMT+1) → 27 Dicembre, 2024 7:00 PM (GMT+1)
    \item \textbf{Is blocking:} SSDGestioneRichiesteVolontariato, SSDControlloCorrettezzaCampi
    \item \textbf{Parent-task:} Casi d'Uso testuali
    \item \textbf{Priority:} Alta
    \item \textbf{Projects:} DoIT-1
    \end{itemize}

Il caso d'uso \textbf{"GestioneRichiesteVolontariato" (UC-003)} descrive l'interazione che c'è tra l'Organizzazione e il Sistema quando questa vuole inserire un'inserzione di una richiesta di volontariato.

L'Incaricato inizia il processo entrando nella schermata di autenticazione a meno che questa non sia già stata fatta. 

L'incaricato una volta effettuato l'autenticazione deve poter inserire nel Sistema la proposta di volontariato e deve comunicare la data, l'orario, il luogo, la categoria, il tipo d'aiuto e infine una descrizione dettagliata di cosa sia richiesto al Volontario.

Una volta inseriti tutti i campi l'Incaricato può procedere all'invio della richiesta. Il Caso d'Uso ``ControlloCorrettezzaCampi'' comunica come viene gestito il flusso alternativo nel caso in cui uno o più campi non siano validi una volta che è stata confermato l'invio della richiesta verso il Sistema.

La richiesta una volta inviata non viene istantaneamente pubblicata sul Sistema, viene momentaneamente bloccata la sua pubblicazione e viene inviata tramite ticket al Comitato di Validazione e Controllo che si deve occupare di accettare o no la richiesta di Volontariato.

Quando la richiesta è stata accettata allora verrà confermata la pubblicazione pubblica sul Sistema.

\paragraph{Informazioni Generali}

\textbf{ID Caso d'Uso:} UC-003
\textbf{Nome:} GestioneRichiesteVolontario
\textbf{Attore Principale:} Organizzazione

\subparagraph{Pre-condizioni}
\begin{enumerate}
    \item L'Organizzazione che intende effettuare l'inserzione possiede i prerequisiti per poter pubblicare un'inserzione
    \item L'Organizzazione è registrata nel Sistema
\end{enumerate}

\subparagraph{Post-condizioni}
\begin{enumerate}
    \item L'inserzione ha tutti i campi correttamente riempiti
    \item L'inserzione viene memorizzata nel Sistema
    \item L'inserzione viene inoltrata al comitato di supervisione
    \item L'inserzione è accettata e viene pubblicata pubblicamente nel Sistema
\end{enumerate}

\paragraph{Scenario Principale}

\subparagraph{Obiettivo}
Un'Organizzazione deve poter pubblicare un'inserzione.

\subparagraph{Flusso Base}
\begin{enumerate}
    \item L'Organizzazione effettua l'autenticazione
    \item L'Organizzazione va nell'apposita schermata di inserimento richieste
    \item Il Sistema mostra l'apposita schermata d'inserimento richieste
    \item In tale pagina l'organizzazione riempie tutti gli opportuni campi del form:
    \begin{enumerate}
        \item Data
        \item Orario
        \item Luogo
        \item Categoria
        \item Tipologia di attività
        \item Descrizione dettagliata di cosa è richiesto ai volontari
        \item Capacità massima dei Volontari che si possono candidare
    \end{enumerate}
    \item L'Organizzazione conferma l'inserimento della richiesta
    \item Il Sistema memorizza correttamente la richiesta
    \item Il Sistema inoltra un ticket al Comitato di Controllo e Validazione per validare la richiesta appena memorizzata
\end{enumerate}

\textit{L'incaricato ripete i punti 2-6 fino a che non indica di aver terminato}

In un secondo momento, non appena il Comitato di Controllo e Validazione ha confermato la richiesta:
\begin{enumerate}
    \item Questa viene ufficialmente salvata nel Sistema
    \item Viene mostrata nell'apposita schermata tramite annuncio o tramite mappa
    \item Un qualsiasi volontario che rispetta i pre-requisiti può inoltrare un'offerta di aiuto
\end{enumerate}

\paragraph{Scenari Alternativi}

\subparagraph{Scenario: Connessione non disponibile}
\begin{enumerate}
    \item Al punto 1 chi ha necessità di inserire un'inserzione deve effettuare l'accesso:
    \begin{enumerate}
        \item L'Organizzazione è già esistente: Ci si riporta al Caso d'Uso: ``RegistraVolontario''
        \item L'Organizzazione non è già esistente:
        \begin{enumerate}
            \item Il Sistema non permette l'inserimento di inserzione senza autenticazione
            \item Un opportuno bottone per uscire dalla pagina di autenticazione è reso disponibile
            \item L'incaricato preme il bottone
            \item Il Caso d'Uso termina
        \end{enumerate}
    \end{enumerate}
    \item Al punto 5 è presente qualche inesattezza nei form: si rimanda al Caso d'Uso: ``ControlloCorrettezzaCampi''
    \item Al punto 8 in caso di rifiuto la gestione è rimandata all'opportuno Caso d'Uso non ancora modellato
\end{enumerate}

\paragraph{Eccezioni e Gestione Errori}

\subparagraph{Errore: Form incompleto}
\textbf{Condizione:} il form riporta irregolarità
\textbf{Azione:} si attiva il Caso d'Uso: ``ControlloCorrettezzaCampi''
\textbf{Risultato:} il Sistema si aspetta che l'utilizzatore sistemi le relative irregolarità, il Sistema quindi non procederà col Caso d'Uso finché queste non saranno sistemate

\paragraph{Requisiti Speciali}

\subparagraph{Requisiti Non Funzionali}
\begin{itemize}
    \item \textbf{Usabilità:} L'interfaccia deve essere intuitiva e accessibile
    \item \textbf{Privacy:} I dati personali del volontario sono condivisi con l'Organizzazione solo dopo l'accettazione dell'offerta
\end{itemize}

\paragraph{Note Aggiuntive}
\begin{itemize}
    \item \textbf{Frequenza:} in base alle necessità, potenzialmente ininterrotto
    \item \textbf{Criticità:} mancanza di connessione
\end{itemize}