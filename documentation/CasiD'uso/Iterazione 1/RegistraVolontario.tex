\title{RegistraVolontario}
\author{Daniele Buser}
\date{Scadenza: 27 Dicembre 2024}
\paragraph{Nota Introduttiva}
\begin{itemize}
\item \textbf{Disciplina:} Requisiti
\item \textbf{Due:} 26 Dicembre, 2024 1:00 AM (GMT+1) → 27 Dicembre, 2024 7:00 PM (GMT+1)
\item \textbf{Is blocking:} RicercaAbbinata, SSDRegistraVolontario
\item \textbf{Parent-task:} Casi d'Uso testuali
\item \textbf{Priority:} Alta
\item \textbf{Projects:} DoIT-1
\end{itemize}

Il caso d'uso \textbf{"Registrazione Nuovo Volontario" (UC-001)} descrive il processo attraverso il quale un nuovo utente può registrarsi come volontario nella piattaforma. Questo processo è considerato di alta priorità essendo una funzionalità core del sistema, con una frequenza prevista di oltre 100 registrazioni mensili.

L'utente inizia il processo dalla schermata iniziale dell'applicazione, dove deve scegliere se accedere come volontario o come organizzazione. Dopo aver selezionato "Volontario", il sistema reindirizza l'utente all'autenticazione Google, un requisito fondamentale per la registrazione.

Una volta fornite le credenziali Google, il sistema verifica automaticamente se esiste già un account associato a quell'utente nel database. Se l'utente è già registrato, viene effettuato automaticamente il login e mostrata la dashboard personale. Se invece l'utente non è presente nel sistema, viene mostrato un form di registrazione specifico per i volontari.

Il sistema importa automaticamente le informazioni base come nome, email e foto profilo dall'account Google e le precompila nel form. L'utente deve quindi fornire informazioni aggiuntive come numero di telefono, data di nascita, città e aree di interesse. Durante la compilazione, il sistema effettua validazioni in tempo reale. Se ci sono campi obbligatori mancanti o compilati in modo scorretto, il sistema impedisce il completamento della registrazione, mostrando messaggi informativi specifici.

Una volta che tutti i dati sono stati inseriti correttamente, l'utente può confermare la registrazione. Il sistema crea l'account e effettua automaticamente il login, reindirizzando l'utente alla dashboard personale dove trova opportunità consigliate e un riepilogo del proprio profilo.

Il processo è stato progettato considerando requisiti non funzionali importanti: la registrazione deve essere completabile in massimo 5 minuti e tutti i dati sensibili devono essere crittografati. Dal punto di vista tecnico, il sistema implementa l'integrazione OAuth con Google, validazione real-time dei dati inseriti e garantisce uno storage sicuro dei dati personali. Per monitorare il corretto funzionamento, il sistema mantiene un logging dettagliato degli errori.

Al termine del processo, il sistema garantisce tre post-condizioni essenziali: la creazione di un nuovo account volontario, l'autenticazione dell'utente nel sistema e la verifica della completezza del profilo.

\paragraph{Informazioni Generali}
\textbf{ID Caso d'Uso:} UC-001

\textbf{Nome:} Registrazione Nuovo Volontario

\textbf{Attore Principale:} Utente non registrato

\textbf{Precondizioni:}
\begin{itemize}
\item Utente non registrato nel sistema come volontario
\item Utente possiede un account Google
\item Dispositivo con connessione internet
\end{itemize}

\textbf{Post-condizioni:}
\begin{itemize}
\item Nuovo account volontario creato
\item Utente autenticato nel sistema
\item Profilo completo e verificato
\end{itemize}

\textbf{Priorità:} Alta

\paragraph{Scenario Principale}

\subparagraph{Obiettivo}
Permettere a un nuovo utente di registrarsi come volontario nella piattaforma utilizzando il proprio account Google e fornendo le informazioni necessarie.

\subparagraph{Flusso Base}
\begin{enumerate}
\item Utente apre l'applicazione
    \begin{itemize}
    \item Sistema: Mostra schermata iniziale
    \end{itemize}
\item Utente seleziona "Volontario"
    \begin{itemize}
    \item Sistema: Reindirizza all'autenticazione Google
    \end{itemize}
\item Utente esegue autenticazione con Google
    \begin{itemize}
    \item Sistema: Verifica esistenza account nel database
    \end{itemize}
\item Se l'utente non esiste:
    \begin{itemize}
    \item Sistema: Mostra form registrazione volontario con:
        \begin{itemize}
        \item Intervallo di date per la disponibilità
        \item Città
        \item Aree di interesse
        \item Descrizione volontario
        \end{itemize}
    \item Sistema: Precompila i dati importati da Google (nome, email, foto)
    \end{itemize}
\item Utente compila e invia form
    \begin{itemize}
    \item Sistema: Salva nuovo account volontario
    \item Sistema: Effettua login automatico
    \end{itemize}
\item Sistema mostra dashboard personale
    \begin{itemize}
    \item Opportunità consigliate
    \item Riepilogo profilo
    \end{itemize}
\end{enumerate}

\paragraph{Scenari Alternativi}

\subparagraph{Scenario: Utente già registrato}
\begin{enumerate}
\item Dopo l'autenticazione Google, sistema rileva account esistente
\item Sistema effettua login automatico
\item Sistema mostra dashboard personale
\end{enumerate}

\paragraph{Eccezioni e Gestione Errori}

\subparagraph{Errore: Form incompleto}
\begin{itemize}
\item \textbf{Condizione:} Campi obbligatori mancanti
\item \textbf{Azione:} Impossibilità a completare registrazione
\item \textbf{Risultato:} Messaggio informativo su campi mancanti o scorretti
\end{itemize}

\subparagraph{Errore: Autenticazione Google fallita}
\begin{itemize}
\item \textbf{Condizione:} Problema con l'autenticazione Google
\item \textbf{Azione:} Interruzione del processo
\item \textbf{Risultato:} Messaggio di errore e possibilità di riprovare
\end{itemize}

\paragraph{Requisiti Speciali}

\subparagraph{Requisiti Non Funzionali}
\begin{itemize}
\item Performance: Registrazione completabile in max 5 minuti
\item Sicurezza: Crittografia dati sensibili
\end{itemize}

\subparagraph{Vincoli Tecnici}
\begin{itemize}
\item Integrazione OAuth Google
\item Validazione real-time
\item Storage sicuro dati personali
\end{itemize}

\paragraph{Note Aggiuntive}
\begin{itemize}
\item \textbf{Frequenza:} Alta (previsione 100+ registrazioni/mese)
\item \textbf{Criticità:} Alta (funzionalità core del sistema)
\item \textbf{Note implementative:}
    \begin{itemize}
    \item Logging dettagliato errori  
    \item Verifica automatica formato dati inseriti
    \item Backup immediato dei dati di registrazione
    \end{itemize}
\end{itemize}