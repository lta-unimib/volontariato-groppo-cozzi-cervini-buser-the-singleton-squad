\title{Gestione Offerte Volontariato}
\author{Gabriele Groppo}
\date{Scadenza: 11 Gennaio 2025}

\paragraph{Nota Introduttiva}
\begin{itemize}
\item \textbf{Disciplina:} Requisiti
\item \textbf{Due:} 11 Gennaio, 2025 1:00 AM (GMT+1) → 11 Gennaio, 2025 1:00 PM (GMT+1)
\item \textbf{Is blocking:} RicercaAbbinata, GestioneOfferteVolontariato
\item \textbf{Parent-task:} Casi d'Uso testuali
\item \textbf{Priority:} Alta
\item \textbf{Projects:} DoIT-2
\end{itemize}

Il caso d'uso \textbf{"GestioneOfferte Volontariato" (UC-004)} descrive l'interazione che c'è tra un Utente Volontario e il Sistema quando questo vuole offrire la propria disponibilità per un'attività di volontariato.

L'Utente Volontario una volta effettuato l'autenticazione deve poter visualizzare nel Sistema le richieste di volontariato disponibili e deve comunicare le sue disponibilità in termini di tempo, competenze, area geografica di intervento.

L'Utente Volontario può offrire la propria disponibilità solo per le attività pubblicate da Organizzazioni verificate nel sistema. Non è possibile offrirsi privatamente per attività non registrate attraverso un'Organizzazione.

L'Utente Volontario deve selezionare una o più competenze tra quelle proposte dall’organizzazione e disponibilità, includendo: fasce orarie precise, e area geografica di azione.

L'Utente Volontario non può offrire la propria disponibilità per più attività che si sovrappongono temporalmente. È però possibile offrirsi per più attività diverse in orari diversi nella stessa giornata.

Il Sistema mostra una mappa interattiva con tutte le richieste di volontariato disponibili, permettendo all'Utente Volontario di interagire direttamente con le Organizzazioni attraverso una chat integrata nella piattaforma.
\paragraph{Informazioni Generali}
\textbf{ID Caso d'Uso}: UC-004 

\textbf{Nome}: Gestisci Offerte Volontariato 
\textbf{Attore Principale}: Utente Volontario

\textbf{Stakeholder:}

\begin{enumerate}
    \item L’Organizzazione ha interesse nel ricevere offerte di volontariato pertinenti alle proprie richieste.
    \item Amministratori del sistema hanno interesse nel matching tra offerte e richieste di volontariato sintomo di un corretto funzionamento dell’sistema stesso.
\end{enumerate}

\textbf{Precondizioni:}

\begin{enumerate}
    \item L'Utente Volontario ha effettuato l'accesso al sistema.
    \item Esistono richieste di volontariato attive nel sistema.
    \item L'Utente Volontario ha completato il proprio profilo con informazioni di base.
\end{enumerate}

\textbf{Post-condizioni:}

\begin{enumerate}
    \item L'offerta di volontariato è registrata nel sistema.
    \item L'Organizzazione riceve una notifica della nuova partecipazione.
    \item Viene creato un canale di chat tra Utente Volontario e Organizzazione.
\end{enumerate}

\textbf{Priorità}: Alta

\paragraph{}{Scenario Principale}

\subparagraph*{Obiettivo}
L'obiettivo del caso d'uso UC-004 "Gestione Offerte Volontariato" è permettere ai volontari di offrire la propria disponibilità per le attività di volontariato pubblicate dalle organizzazioni nel sistema, assicurando un matching efficace tra le competenze/disponibilità offerte e le necessità richieste.

\subparagraph*{Flusso Base}

\begin{enumerate}
    \item Il Sistema mostra all'Utente Volontario una lista con le richieste di volontariato disponibili sulla base delle sue preferenze (posizione, tipologia di volontariato, …).
    \item L'Utente Volontario visualizza l'elenco delle richieste di volontariato filtrabili per:
    \begin{itemize}
        \item Tipo di supporto richiesto
        \item Zona geografica
        \item Data e fascia oraria
        \item Competenze richieste
    \end{itemize}
    \item L'Utente Volontario seleziona una richiesta di interesse.
    \item Il Sistema mostra i dettagli completi della richiesta selezionata.
    \item L'Utente Volontario indica la propria disponibilità per la richiesta selezionata specificando tra le proposte dall’Organizzazione:
    \begin{itemize}
        \item Le fasce orarie
        \item Il tipo specifico di supporto che intende offrire
        \item L'area geografica in cui è disponibile ad operare
    \end{itemize}
    \item Il Sistema verifica la compatibilità tra disponibilità offerta e richiesta.
    \item Il Sistema notifica l'Organizzazione della nuova offerta di volontariato.
    \item L'Utente Volontario può iniziare una chat con l'Organizzazione per eventuali chiarimenti.
\end{enumerate}

\paragraph{section}{Scenari Alternativi}

\subparagraph*{Scenario: L’utente consulta la mappa}
\begin{enumerate}
    \item L’utente decide di consultare le richieste di volontariato attraverso la mappa.
    \begin{itemize}
        \item Punto 2 scenario principale.
    \end{itemize}
\end{enumerate}

\paragraph{Eccezioni e Gestione Errori}

\subparagraph*{Errore: Connessione assente}
\begin{itemize}
    \item \textbf{Condizione}: Il dispositivo dell'utente perde la connessione durante l'interazione col sistema.
    \item \textbf{Azione}: Il sistema mostra un messaggio di errore all’utente.
    \item \textbf{Risultato}: L’inserimento dell’offerta di volontariato viene annullato.
\end{itemize}

\subparagraph*{Errore: Dati incompleti}
\begin{itemize}
    \item \textbf{Condizione}: Il volontario tenta di inviare l'offerta senza tutti i dati obbligatori.
    \item \textbf{Azione}: Il sistema verifica la completezza dei campi richiesti.
    \item \textbf{Risultato}: Vengono evidenziati i campi mancanti con richiesta di compilazione.
\end{itemize}

\subparagraph*{Eccezione: Richiesta già presa}
\begin{itemize}
    \item \textbf{Condizione}: Il volontario tenta di inviare l'offerta ma il server risponde che quella richiesta di volontariato è già stata soddisfatta.
    \item \textbf{Azione}: Il sistema mostra un messaggio di errore.
    \item \textbf{Risultato}: L’inserimento dell’offerta di volontariato viene annullato.
\end{itemize}

\paragraph{Requisiti Speciali}

\subparagraph*{Requisiti Non Funzionali}

\begin{itemize}
    \item \textbf{Usabilità}: L'interfaccia deve essere intuitiva e accessibile.
    \item \textbf{Privacy}: I dati personali del volontario sono condivisi con l'Organizzazione solo dopo l'accettazione dell'offerta.
\end{itemize}

\subparagraph*{Vincoli Tecnici}

\begin{itemize}
    \item \textbf{Performance}: Il sistema deve mostrare la mappa e le richieste entro 3 secondi.
\end{itemize}

\paragraph{Note Aggiuntive}

\begin{itemize}
    \item \textbf{Frequenza}: Il caso d’uso viene eseguito ogni volta che un utente si offre per effettuare volontariato.
    \item \textbf{Criticità}: Alta, in quanto è una delle attività di business principali del sistema software.
    \item \textbf{Note implementative}: Cercare di automatizzare il più possibile le operazioni di matching tra richieste e offerte in modo da non lasciare spazio ad ambiguità.
\end{itemize}