\title{Specifica Supplementare}
\author{Matteo Cervini \and Andrea Cozzi \and Gabriele Groppo \and Daniele Buser}
\date{Scadenza: 11 Gennaio, 2025}

\maketitle

\subsubsection{Nota Introduttiva}
\begin{itemize}
\item \textbf{Disciplina:} Requisiti
\item \textbf{Due:} 11 Gennaio, 2025 1:00 AM (GMT+1) → 11 Gennaio, 2025 7:00 PM (GMT+1)
\item \textbf{Priority:} Medio
\item \textbf{Projects:} DoIT-2

\end{itemize}
\subsubsection{Requisiti Funzionali}

\paragraph{Registrazione e Gestione degli Utenti}

\begin{enumerate}
    \item Gli utenti devono poter creare uno o entrambi gli account scegliendo tra i seguenti profili:
    \begin{itemize}
        \item Volontario singolo
        \item Associazione o organizzazione di volontariato o beneficiario singolo
    \end{itemize}
    
    \item Devono essere richieste informazioni obbligatorie per la registrazione, quali:
    \begin{itemize}
        \item Nome e cognome (o ragione sociale per aziende/enti)
        \item Email
        \item Password sicura con requisiti minimi di complessità
    \end{itemize}
    
    \item Ogni utente deve avere la possibilità di completare un profilo personalizzato, comprensivo di:
    \begin{itemize}
        \item Disponibilità orarie
        \item Competenze specifiche
        \item Località
    \end{itemize}
\end{enumerate}

\paragraph{Gestione delle Richieste di Volontariato}

\begin{enumerate}
    \item I beneficiari devono poter creare richieste di supporto specificando:
    \begin{itemize}
        \item Categoria di aiuto
        \item Descrizione dettagliata della necessità
        \item Data e orario preferiti
        \item Localizzazione geografica tramite mappa interattiva
    \end{itemize}
    
    \item Le richieste devono poter essere modificate o cancellate fino a quando non vengono accettate da un volontario.
\end{enumerate}

\paragraph{Gestione delle Offerte di Volontariato}

\begin{enumerate}
    \item I volontari devono poter indicare la propria disponibilità attraverso:
    \begin{itemize}
        \item Calendario settimanale con fasce orarie selezionabili
        \item Tipologia di attività offerte
        \item Zone geografiche preferite
    \end{itemize}
    
    \item Devono poter modificare o aggiornare le proprie disponibilità in qualsiasi momento.
\end{enumerate}

\paragraph{Sistema di Abbinamento (Matching)}

\begin{enumerate}
    \item La piattaforma deve includere un algoritmo di matching che:
    \begin{itemize}
        \item Confronti le richieste dei beneficiari con le disponibilità dei volontari
        \item Utilizzi parametri quali categoria, localizzazione e orario per proporre abbinamenti
    \end{itemize}
    
    \item Gli utenti devono ricevere notifiche per ogni proposta di abbinamento rilevante.
\end{enumerate}

\paragraph{Sistema di Messaggistica}

\begin{enumerate}
    \item Deve essere integrato un sistema di chat sicuro per consentire la comunicazione tra volontari e beneficiari.
    
    \item Le funzionalità principali della chat includono:
    \begin{itemize}
        \item Invio di messaggi testuali
        \item Notifiche in tempo reale
        \item Cronologia dei messaggi
    \end{itemize}
\end{enumerate}

\paragraph{Sistema di Valutazione e Feedback}

\begin{enumerate}
    \item Dopo il completamento di un'attività, entrambe le parti devono poter lasciare una valutazione.
    
    \item Le recensioni devono includere:
    \begin{itemize}
        \item Punteggio da 1 a 5 stelle
        \item Commento opzionale
    \end{itemize}
    
    \item I punteggi devono essere visibili sui profili pubblici degli utenti.
\end{enumerate}

\paragraph{Comitato di Controllo e Validazione}

\begin{enumerate}
    \item La piattaforma deve consentire a un team dedicato di:
    \begin{itemize}
        \item Verificare le richieste e le offerte inserite dagli utenti
        \item Segnalare e rimuovere contenuti inappropriati
    \end{itemize}
    
    \item Deve essere previsto un sistema per ricevere e gestire segnalazioni dagli utenti.
\end{enumerate}

\paragraph{Integrazione con Social Media}

\begin{enumerate}
    \item Gli utenti devono poter condividere richieste e offerte di volontariato sui principali social network.
\end{enumerate}

\subsubsection{Requisiti Non Funzionali}

\paragraph{Prestazioni}

\begin{enumerate}
    \item Il sistema deve garantire tempi di risposta inferiori a 2 secondi per il caricamento delle principali pagine e operazioni.
    \item L'algoritmo di matching deve essere in grado di elaborare richieste e offerte in meno di 5 secondi.
    \item Aggiornamento in tempo reale delle notifiche di chat
\end{enumerate}

\paragraph{Scalabilità}

\begin{enumerate}
    \item Deve essere possibile aumentare la capacità del sistema con l'aumento del numero di utenti.
\end{enumerate}

\paragraph{Sicurezza}

\begin{enumerate}
    \item I dati degli utenti devono essere protetti attraverso crittografia sia a riposo che in transito.
\end{enumerate}

\paragraph{Compatibilità}

\begin{enumerate}
    \item L'app deve essere compatibile con:
    \begin{itemize}
        \item iOS
        \item Android
    \end{itemize}
    \item Se previsto un portale web, deve essere ottimizzato per i principali browser.
\end{enumerate}

\paragraph{Usabilità}

\begin{enumerate}
    \item L'interfaccia deve essere intuitiva e accessibile
    \item Deve essere garantita un'esperienza coerente su tutti i dispositivi
    \item Il processo di offerta deve essere completabile in pochi passaggi intuitivi
\end{enumerate}

\paragraph{Manutenibilità}

\begin{enumerate}
    \item Il codice deve essere scritto seguendo standard di programmazione modulari
    \item Deve essere garantito il supporto per aggiornamenti regolari della piattaforma
\end{enumerate}

\paragraph{Disponibilità e Affidabilità}

\begin{enumerate}
    \item La piattaforma deve essere operativa almeno il 99,5\% del tempo su base mensile.
\end{enumerate}

\paragraph{Privacy}

\begin{enumerate}
    \item I dati personali del volontario devono essere condivisi con l'organizzazione solo dopo l'accettazione dell'offerta
    \item Le conversazioni nella chat devono essere protette e accessibili solo alle parti coinvolte
\end{enumerate}

\subsubsection{Requisiti Hardware}

\paragraph{Server e Servizi di Hosting}

Il progetto DoIT utilizzerà una soluzione di hosting cloud per garantire scalabilità e affidabilità:

\begin{itemize}
    \item \textbf{Cloud Provider}: Render è usato per hostare il server
    \item \textbf{Database}:
    \begin{itemize}
        \item Neon Database hosta il database remoto
        \item PostgreSQL per dati strutturati
    \end{itemize}
\end{itemize}

\paragraph{Esigenze di Storage}

Per la gestione dei dati, il sistema utilizzerà:

\begin{itemize}
    \item PostgreSQL per dati strutturati
    \item Neon per messaggistica
    \item Neon per file utente
\end{itemize}

\subsubsection{Requisiti Software}

\paragraph{Tecnologie Preferite per lo Sviluppo}

\begin{itemize}
    \item \textbf{Frontend}:
    \begin{itemize}
        \item HTML, CSS e TailwindCSS
        \item TypeScript e Next.js
    \end{itemize}
    
    \item \textbf{Backend}:
    \begin{itemize}
        \item Java con Spring Boot
        \item Spring Security
        \item Spring Data JPA
        \item OpenAPI
    \end{itemize}
    
    \item \textbf{Database}:
    \begin{itemize}
        \item PostgreSQL
        \item Flyway
    \end{itemize}
    
    \item \textbf{Tools/Utilities}:
    \begin{itemize}
        \item Git
        \item Maven
        \item npm
        \item SonarQube
        \item Understand
        \item Mermaid Live e Lucidchart
    \end{itemize}
\end{itemize}

\paragraph{Integrazioni con Altre Piattaforme o API Esterne}

\begin{itemize}
    \item Geolocalizzazione (Google Maps o Mapbox)
    \item Social Media (Facebook, Twitter, LinkedIn)
\end{itemize}

\subsubsection{Specifiche di Interfaccia}

\paragraph{Requisiti per il Design dell'Interfaccia Utente}

\begin{enumerate}
    \item \textbf{Design Responsivo}:
    \begin{itemize}
        \item Interfaccia completamente responsive
        \item Utilizzo di TailwindCSS
    \end{itemize}
    
    \item \textbf{Estetica Pulita e Semplice}:
    \begin{itemize}
        \item Design minimale
        \item Colori chiari e contrastanti
        \item Interfaccia coerente
    \end{itemize}
    
    \item \textbf{Navigazione Intuitiva}:
    \begin{itemize}
        \item Barra di navigazione chiara
        \item Funzionalità di ricerca avanzata
        \item Sistema di notifiche visibile
    \end{itemize}
    
    \item \textbf{Iconografia Chiara}:
    \begin{itemize}
        \item Icone comprensibili
        \item Stile uniforme
    \end{itemize}
    
    \item \textbf{Accessibilità}:
    \begin{itemize}
        \item Conformità WCAG 2.1
        \item Supporto screen reader
        \item Font leggibili
    \end{itemize}
\end{enumerate}

\paragraph{Linee Guida per l'Esperienza Utente (UX)}

\begin{enumerate}
    \item \textbf{Semplicità e Chiarezza}
    \item \textbf{Flusso Intuitivo}
    \item \textbf{Feedback Visivo e Interattivo}
    \item \textbf{Personalizzazione}
    \item \textbf{Chiarezza nelle Interazioni}
    \item \textbf{Performance e Velocità}
\end{enumerate}

\subsubsection{Vincoli di Progetto}

\paragraph{Budget disponibile per sviluppo e manutenzione}

\begin{itemize}
    \item Progetto universitario senza budget specifico
    \item Sviluppo da parte di team accademico
\end{itemize}

\paragraph{Tempi di rilascio per MVP e versioni successive}

\begin{itemize}
    \item \textbf{MVP}: Completamento entro 1 mese
    \item \textbf{Versioni successive}: Cicli mensili di aggiornamento
\end{itemize}

\subsubsection{Gestione del Rischio}

\paragraph{Rischi Aggiuntivi}

\begin{enumerate}
    \item \textbf{Mancanza di partecipazione da parte degli utenti}
    \item \textbf{Problemi di sicurezza e privacy dei dati}
    \item \textbf{Difficoltà tecniche nell'integrazione}
    \item \textbf{Scalabilità limitata nell'MVP}
    \item \textbf{Ritardi nello sviluppo}
\end{enumerate}

\paragraph{Strategie per Mitigare i Rischi}

\begin{enumerate}
    \item \textbf{Strategia per l'integrazione tecnica}:
    \begin{itemize}
        \item Seguire guide ufficiali
        \item Creare prototipi
    \end{itemize}
    
    \item \textbf{Strategia per la scalabilità}:
    \begin{itemize}
        \item Architettura modulare
        \item Load testing
    \end{itemize}
    
    \item \textbf{Strategia per i ritardi}:
    \begin{itemize}
        \item Pianificazione agile
        \item Valutazione risorse aggiuntive
    \end{itemize}
\end{enumerate}