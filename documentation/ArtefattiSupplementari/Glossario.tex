\title{Specifica Supplementare}
\author{Matteo Cervini \and Andrea Cozzi \and Gabriele Groppo \and Daniele Buser}
\date{Scadenza: 11 Gennaio, 2025}

\maketitle

\subsubsection{Nota Introduttiva}
\begin{itemize}
\item \textbf{Disciplina:} Requisiti
\item \textbf{Due:} 11 Gennaio, 2025 1:00 AM (GMT+1) → 11 Gennaio, 2025 7:00 PM (GMT+1)
\item \textbf{Priority:} Medio
\item \textbf{Projects:} DoIT-2
\end{itemize}

\subsubsection{Glossario Tecnico}

\begin{itemize}

\item \textbf{Autenticazione}

Processo di verifica dell'identità dell'utente effettuato tramite email e password. È un passaggio obbligatorio durante la registrazione che garantisce la sicurezza e l'unicità dell'account.

\item \textbf{Crittografia}

Tecnica di sicurezza utilizzata per proteggere i dati sensibili degli utenti nel sistema. È un requisito non funzionale che garantisce la protezione delle informazioni personali memorizzate.

\item \textbf{Dashboard Personale}

Interfaccia utente principale mostrata dopo il login, contenente:
\begin{itemize}
    \item Opportunità di volontariato consigliate
    \item Riepilogo del profilo utente
    \item Altre funzionalità personalizzate
\end{itemize}

\item \textbf{Form di Registrazione}

Modulo di registrazione che raccoglie informazioni specifiche del volontario. Include:
\begin{itemize}
    \item Campi obbligatori da compilare
    \item Validazione in tempo reale
\end{itemize}

\item \textbf{Login Automatico}

Processo che avviene dopo la registrazione completata con successo, permettendo l'accesso immediato alla dashboard personale senza necessità di inserire nuovamente le credenziali.

\item \textbf{Opportunità}

Attività di volontariato disponibili sulla piattaforma che vengono:
\begin{itemize}
    \item Consigliate ai volontari sulla dashboard
    \item Filtrate in base al profilo dell'utente e area geografica
    \item Presentate dopo la registrazione
    \item Pubblicate solo da Organizzazioni verificate
    \item Non sovrapponibili temporalmente per lo stesso volontario
\end{itemize}

\item \textbf{Organizzazione}

Tipologia di account alternativa a quella del volontario, con funzionalità e processo di registrazione specifici.

\item \textbf{Profilo}

Insieme delle informazioni che caratterizzano il volontario, composto da:
\begin{itemize}
    \item Informazioni aggiuntive fornite durante la registrazione
    \item Stato di completezza e verifica
\end{itemize}

\item \textbf{Registrazione}

Processo di creazione di un nuovo account volontario che include:
\begin{itemize}
    \item Verifica dei prerequisiti (account Google)
    \item Raccolta dati tramite form
    \item Validazione in tempo reale
    \item Tempo massimo di completamento: 5 minuti
\end{itemize}

\item \textbf{Storage Sicuro}

Sistema di memorizzazione dei dati personali che implementa:
\begin{itemize}
    \item Crittografia delle informazioni sensibili
    \item Protezione dei dati personali
    \item Conformità con i requisiti di sicurezza
\end{itemize}

\item \textbf{Validazione Real-time}

Sistema di controllo immediato dei dati inseriti che:
\begin{itemize}
    \item Verifica la correttezza delle informazioni
    \item Mostra messaggi di errore specifici
    \item Impedisce la sottomissione di dati non validi
\end{itemize}

\item \textbf{Volontario}

Utente registrato nella piattaforma che:
\begin{itemize}
    \item Ha completato il processo di registrazione
    \item Può accedere alle opportunità di volontariato
    \item Può offrire disponibilità per attività non sovrapposte temporalmente
    \item Deve specificare competenze e disponibilità temporali
\end{itemize}

\item \textbf{Area Geografica}

Zona territoriale in cui il volontario è disponibile ad operare, specificata durante l'offerta di disponibilità per un'attività di volontariato.

\item \textbf{Chat Integrata}

Sistema di messaggistica interno alla piattaforma che permette la comunicazione diretta tra Volontari e Organizzazioni dopo l'offerta di disponibilità.

\item \textbf{Competenze}

Capacità e abilità specifiche richieste dalle Organizzazioni e selezionabili dai Volontari durante l'offerta di disponibilità per un'attività.

\item \textbf{Mappa Interattiva}

Visualizzazione geografica delle richieste di volontariato disponibili che permette ai volontari di:
\begin{itemize}
    \item Visualizzare la distribuzione delle opportunità sul territorio
    \item Interagire direttamente con le richieste
    \item Filtrare le attività per zona
\end{itemize}

\item \textbf{Organizzazione Verificata}

Status speciale assegnato alle organizzazioni che hanno completato un processo di verifica nel sistema, requisito necessario per pubblicare richieste di volontariato.

\item \textbf{Controllo Duplicato Account}

Sistema che verifica l'esistenza di un account già registrato associato alla stessa organizzazione, impedendo la duplicazione degli account e fornendo all'utente le alternative per risolvere eventuali conflitti.

\item \textbf{Errore di Registrazione}

Messaggio che il sistema mostra quando il processo di registrazione non può essere completato, generalmente dovuto a dati mancanti, errati o duplicati. L'errore fornisce solitamente dettagli specifici per correggere il problema.

\item \textbf{Conferma di Registrazione}

La fase finale del processo di registrazione in cui l'organizzazione approva i dati inseriti e conferma la creazione del proprio account, permettendo al sistema di procedere con il login automatico e il reindirizzamento alla dashboard.

\item \textbf{Piattaforma}

Il sistema complessivo che include sia le funzionalità per i volontari che quelle per le organizzazioni, come la gestione dei profili, l'accesso alle opportunità di volontariato e le interazioni tra le diverse entità.

\end{itemize}