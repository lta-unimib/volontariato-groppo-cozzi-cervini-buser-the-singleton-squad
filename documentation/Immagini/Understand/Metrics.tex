Le metriche individuate da Understand sul package \textbf{doit} sono le seguenti:\\
\fbox{\begin{minipage}{\textwidth}
    \begin{itemize}
        \item \textbf{Blank Lines:} 759
        \item \textbf{Classes:} 113
        \item \textbf{Code Lines:} 3529
        \item \textbf{CodeCheck Violation Density by Code Lines:} 0.00
        \item \textbf{CodeCheck Violation Density by Lines:} 0.00
        \item \textbf{CodeCheck Violation Types:} 0
        \item \textbf{CodeCheck Violation Types:} 0
        \item \textbf{Comment Lines:} 160
        \item \textbf{Comments to Code Ratio:} 0.04
        \item \textbf{Declarative Statements:} 1717
        \item \textbf{Executable Statements:} 879
        \item \textbf{Executable Statements:} 107
    \end{itemize}
\end{minipage}}\\[2ex]

Questa analisi ci comunica che abbiamo un assenza di violazioni, densità: 0.00.
Ciò significa che il codice è ben strutturato e segue standard definiti,
113 classi, 1717 dichiarazioni, e 879 istruzioni eseguibili indicano un codice con una complessità moderata.
Il rapporto commenti/codice di 0.04 è leggermente basso ma comquneque accettabile.
Il codice è ben strutturato e non presenta violazioni di standard, il che è fondamentale per la manutenibilità e 
la qualità del software.
